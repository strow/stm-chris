% Created 2020-09-30 Wed 15:22
% Intended LaTeX compiler: pdflatex
\documentclass[10pt,t]{beamer}
\usepackage[utf8]{inputenc}
\usepackage{graphicx}
\usepackage{grffile}
\usepackage{longtable}
\usepackage{wrapfig}
\usepackage{rotating}
\usepackage{textcomp}
\usepackage{amssymb}
\usepackage{capt-of}
\usepackage{hyperref}
\usetheme{default}
%
\usepackage[font=small,labelfont=bf]{caption} % Required for specifying captions
%
\author{C. L. Hepplewhite}
\date{\today}
\title{\large Radiative Transfer Algorithm Updates}
\subtitle{\footnotesize{AIRS Virtual Science Team Meeting}}
\date{\vspace{0.1in}\footnotesize{October 2020 \vfill}}
\author{C. L. Hepplewhite\inst{1,2}, L. Larrabee Strow\inst{1,2}, and S. deSouza Machado\inst{1,2} }
\institute[UMBC]{\inst{1} UMBC Physics Dept. \and \inst{2}UMBC JCET}
\input beamer_setup
\metroset{titleformat title=allcaps}
\setbeamertemplate{frame footer}{UMBC Atmospheric Spectroscopy Lab}
\begin{document}

\maketitle

% -----------------------------------------------------
\begin{frame}{Summary}
\begin{itemize}
  \item What is the Stand-alone radiative transfer algorithm (SARTA)
  \item Who uses SARTA.
  \item Current SARTA build status.
  \item Plans.
    
\end{itemize}

\end{frame}
% -----------------------------------------------------
\begin{frame}{The SARTA}

  \begin{itemize}
  \item The Stand-alone radiative transfer algorithm (SARTA) is constructed using kCARTA
  \item Therefore SARTA has the same spectroscopy as kCARTA.
  \item SARTA was developed 18 years ago for the AIRS.
  \item It uses sets of coefficients that parameterize atmospheric transmittances derived using a set of training profiles.
  \item Is written in Fortran
  \item Permits very fast computation of radiances for predefined spectral response functions.
    \item Has a version for clear sky radiance calculations and for cloudy radiances.
    
\end{itemize}

\end{frame}
% ----------------------------------------------------
\begin{frame}{Who uses SARTA}

  \begin{itemize}
  \item SARTA is used to compute clear and all-sky radiances for any and all FoVs from AIRS, CrIS and IASI missions.
  \item Is fast enough to make whole-mission modelling easily manageable. Faster than kCARTA *way* faster than LBL.
  \item Currently used in ASL for the RTP production for analysis of sensor performance and global studies and geophysical retireval.
  \item Is used in the AIRS geophysical product retrieval.
    \item Is used in NUCAPS.

  \end{itemize}
\end{frame}
% ----------------------------------------------------
\begin{frame}{Current Spectroscopy}

  \begin{itemize}
  \item HITRAN 2016
  \item CO2, CH4 line mixing from LBLRTM12.8
  \item MT CKD3.2
  \item CO2 CIA from WV and N2 by Hartmann (4.3 um)
  \item Single parameter surface emissivity.
    

  \end{itemize}
\end{frame}
% ---------------------------------------------------
\begin{frame}{Future Improvements}

  \begin{itemize}
  \item HITRAN 2020.
  \item Line Mixing package from the HITRAN folks (Iouli Gordon).
  \item Currently use kCARTA at 0.0025 cm-1, can update to 0.0005 cm-1 in 15 um region
  \item poss. linear-in-tau RT.
  \item poss. Nalli surface emissivity parameterization.
  \item poss. look into running off NLTE from Manuel esp. the extreme solar angles.
  \item Tuning for 'real world' application.
      
  \end{itemize}
\end{frame}

% -------------------------------------------------
\begin{frame}{Current SARTAs at ASL}

  \begin{itemize}
  \item The following SARTAs are in use at ASL:
    \begin{itemize}
    \item AIRS vL1B (2008) as supplied to JPL is tuned.
    \item AIRS vL1C (2016) as supplied untuned.
    \item CrIS NSR v2008, \& v2016.
    \item CrIS FSR v2016 (untuned).
    \item IASI v2008 and v2016 (untuned).
    \item CHIRP v2016 (under development).
    \end{itemize}
  \end{itemize}
  

\end{frame}
% ------------------------------------------------
\begin{frame}{Validating by bias and residual}

  \begin{itemize}
  \item After completing the fast coefficient regression, top-of-atmosphere (TOA) radiances predicted by SARTA are compared to those from kCARTA from the training set or extended profile set.
  \end{itemize}
      \begin{center}
    \includegraphics[width=0.6\linewidth]{./Figs/chirp_49regr_sar_kc_bias_stdv.pdf}
    \captionof{figure}{CHIRP channels. AIRS bias relative to SNPP from global statistics. Showing AIRS module bands.}
  \end{center}
  

\end{frame}
% -----------------------------------------------
\begin{frame}{Improved bias \& std.dev relative to kCARTA}

  \begin{itemize}
  \item kCARTA monochromatic layer-so-space optical depths are convolved to the sensor grid.
  \item For AIRS (and any spectrometer) convolution with the spectral response functions (SRF) is well behaved. Transmittance approaches zero for opaque channels.
  \item For interferometers (IASI, CrIS, CHIRP) convolution with the instrument line shape (ILS) produces tau that goes -ve and +ve into the opaque region.
  \item Regression of the fast coefficients is controlled down to a minimum transmittance which is set by inspection of these 'wings'.
  \item Sarta builds for IASI, CrIS version 2019 and CHIRP 2020 included some fits in strong (CO2) bands that were not optimal. (See next slides).
  \end{itemize}

\end{frame}

% -----------------------------------------------
\begin{frame}{Example of optimization: CHIRP 640 cm-1}

  \begin{center}
    \includegraphics[width=0.6\linewidth]{./Figs/chirp_optimize1.png}
    \captionof{figure}{CHIRP SARTA bias compared to kCARTA, with and without improved regression.}
  \end{center}


\end{frame}
% -----------------------------------------------
\begin{frame}{Example of optimization: IASI 4.3 um}

\vspace{-0.5cm}
  \begin{center}
    \includegraphics[width=0.6\linewidth]{./Figs/iasi_sarta_kcarta_mean_bias_4p3um_fx.png}
    \captionof{figure}{IASI SARTA bias compared to kCARTA, with and without improved regression at 4.3 um.}
  \end{center}

\end{frame}
% -----------------------------------------------
\begin{frame}{Tuning: 1}

  \begin{block}{}
    In the real world differences in measured top-of-atmosphere radiances compared to calculated have several potential sources:
  \end{block}
    
  \begin{itemize}
  \item AIRS radiometric calibration.
  \item AIRS spectral calibration and instrument line shape.
  \item AIRS fast model parameterization.
  \item cloud contamination in fields of view selected as clear.
  \item validation data, including time/space mismatches and uncertainties in minor gas abundances.
    \item spectroscopy used in the AIRS RTA.

  \end{itemize}

\end{frame}
% -----------------------------------------------
\begin{frame}{Tuning: 2}

  \begin{block}{}
    The objective of tuning is to reduce bias in the real world sufficiently to increase the yield in single footprint geophysical retrievals.
  \end{block}

  \begin{itemize}
  \item Large data sets are compared covering all global atmospheric and scene types.
  \item Bias and std.dev characterisctics between coincident TOA measurements with forward model predictions are determined.
  \item RTA transmittances are adjusted for specific channels where needed.
  \item SARTA uses a supporting data file of tuning parameters.
  \item Tuning of transmittances in the RTA are preferred over tuning BT of TOA predicts.

  \end{itemize}

\end{frame}
% ----------------------------------------------
\begin{frame}{Tuning: Case study}

  \begin{block}{}
  \end{block}

  \begin{itemize}
  \item Selection of June 25 talk
  \end{itemize}

\end{frame}
% ----------------------------------------------
\begin{frame}{Conclusions}

  \begin{block}{}
  \end{block}

  \begin{itemize}
  \item RTA (kCARTA and SARTA) development and maintenance continues at UMBC/JCET.
  \item SARTA for AIRS, CrIS (NSR FSR), IASI and CHIRP have been or are being developed, tested
    and verified.
  \item CHIRP SARTA is in progress.
  \item Tuning for builds from 2019 are either partial or pending.
    
  \end{itemize}

\end{frame}
% -----------------------------------------------


\end{document}
