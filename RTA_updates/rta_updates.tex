% Created 2020-09-30 Wed 15:22
% Intended LaTeX compiler: pdflatex
\documentclass[10pt,t]{beamer}
\usepackage[utf8]{inputenc}
\usepackage{graphicx}
\usepackage{grffile}
\usepackage{longtable}
\usepackage{wrapfig}
\usepackage{rotating}
\usepackage{textcomp}
\usepackage{amssymb}
\usepackage{capt-of}
\usepackage{hyperref}
\usetheme{default}
%
\usepackage[font=small,labelfont=bf]{caption} % Required for specifying captions
%
\author{C. L. Hepplewhite}
\date{\today}
\title{\large Radiative Transfer Algorithm Updates}
\subtitle{\footnotesize{AIRS Virtual Science Team Meeting}}
\date{\vspace{0.1in}\footnotesize{October 2020 \vfill}}
\author{C. L. Hepplewhite\inst{1,2}, L. Larrabee Strow\inst{1,2}, and S. deSouza Machado\inst{1,2} }
\institute[UMBC]{\inst{1} UMBC Physics Dept. \and \inst{2}UMBC JCET}
\input beamer_setup
\metroset{titleformat title=allcaps}
\setbeamertemplate{frame footer}{UMBC Atmospheric Spectroscopy Lab}
\begin{document}

\maketitle

% -----------------------------------------------------
\begin{frame}{Summary}
\begin{itemize}
  \item What is the Stand-alone radiative transfer algorithm (SARTA)
  \item Who uses SARTA.
  \item Current SARTA build status.
  \item Plans.
    
\end{itemize}

\end{frame}
% -----------------------------------------------------
\begin{frame}{The SARTA}

  \begin{itemize}
  \item The Stand-alone radiative transfer algorithm (SARTA) is constructed using kCARTA
  \item Therefore SARTA has the same spectroscopy as kCARTA.
  \item SARTA was developed 18 years ago for the AIRS.
  \item It uses sets of coefficients that parameterize atmospheric transmittances derived using a set of training profiles.
  \item Is written in Fortran
  \item Permits very fast computation of radiances for predefined spectral response functions.
    \item Has a version for clear sky radiance calculations and for cloudy radiances.
    
\end{itemize}

\end{frame}
% ----------------------------------------------------
\begin{frame}{Who uses SARTA}

  \begin{itemize}
  \item SARTA is used to compute clear and all-sky radiances for any and all FoVs from AIRS, CrIS and IASI missions.
  \item Is fast enough to make whole-mission modelling easily manageable.
  \item Currently used in ASL for the RTP production for analysis of sensor performance and global studies and geophysical retireval.
  \item Is used in the AIRS geophysical product retrieval.
    \item Is used in NUCAPS.

  \end{itemize}
\end{frame}
% ----------------------------------------------------
\begin{frame}

  \begin{itemize}
  \item One

  \end{itemize}
\end{frame}
% ---------------------------------------------------

\end{document}
